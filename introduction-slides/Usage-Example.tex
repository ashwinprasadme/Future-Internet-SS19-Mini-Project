%%%%%%%%%%%%%%%%%%%%%%%%%%%%%%%%%%%%%%%%%%%%%%%
% University Paderborn Beamer Presentation 

% Author: Ashwin Prasad Shivarpatna Venkatesh 

%This template is free: you can redistribute it and/or modify
%it under the terms of the GNU General Public License as published by
%the Free Software Foundation, either version 3 of the License, or any later version.
%
%This program is distributed in the hope that it will be useful,
%but WITHOUT ANY WARRANTY; without even the implied warranty of
%MERCHANTABILITY or FITNESS FOR A PARTICULAR PURPOSE.  See the
%GNU General Public License for more details.
%
%You should have received a copy of the GNU General Public License
%along with this program.  If not, see <https://www.gnu.org/licenses/>.

%%%%%%%%%%%%%%%%%%%%%%%%%%%%%%%%%%%%%%%%%%%%%%%

\documentclass{beamer}
% Default page size 12.8cm x 9.6cm

% import packages and user-defined commands
\input{internal/packages}
\input{internal/commands}
\input{internal/upb-unicolors}
\input{internal/upb-theme}
% Title
\title{Complex Service Orchestration with OSM and OpenStack} 

% Sub Title
\subtitle{Mini-Project Intermediate Presentation}

% Your name
\author{Ashwin Prasad Shivarpatna Venkatesh}

% Your institution for the title page
\institute{Future Internet - SS19} 

% Date, can be changed to a custom date
\date{\today} 

% Choose primary UPB color for title, headings etc.. 
% Choose from 
% (uni-cyan, uni-black, uni-blue, uni-orange, uni-purple, uni-green)
\newcommand{\upbcolor}{uni-blue} 




%----------------------------------------------------------------------------------------
%	TITLE PAGE
%----------------------------------------------------------------------------------------


\begin{document}

{
\upbtitlebackground 
\begin{frame}
%\upblogo
\titlepage % Print the title page as the first slide
\end{frame}
}

\begin{frame}
\frametitle{Overview} % Table of contents slide, comment this block out to remove it
\tableofcontents % Throughout your presentation, if you choose to use \section{} and \subsection{} commands, these will automatically be printed on this slide as an overview of your presentation
\end{frame}

%----------------------------------------------------------------------------------------
%	PRESENTATION SLIDES
%----------------------------------------------------------------------------------------

%------------------------------------------------
\section{First Section} % Sections can be created in order to organize your presentation into discrete blocks, all sections and subsections are automatically printed in the table of contents as an overview of the talk
%------------------------------------------------

\subsection{Subsection Example} % A subsection can be created just before a set of slides with a common theme to further break down your presentation into chunks

\begin{frame}
\frametitle{Paragraphs of Text}

Paragraphs Sed iaculis dapibus gravida. Morbi sed tortor erat, nec interdum arcu. Sed id lorem lectus. Quisque viverra augue id sem ornare non aliquam nibh tristique. Aenean in ligula nisl. Nulla sed tellus ipsum. Donec vestibulum ligula non lorem vulputate fermentum accumsan neque mollis.\\~\\

Sed diam enim, sagittis nec condimentum sit amet, ullamcorper sit amet libero. Aliquam vel dui orci, a porta odio. Nullam id suscipit ipsum. 
\end{frame}

%------------------------------------------------

\begin{frame}
\frametitle{Bullet Points}
\begin{itemize}[<+->]
\item Lorem ipsum dolor sit amet, consectetur adipiscing elit
\begin{itemize}
\item Aliquam blandit faucibus nisi, sit amet dapibus enim tempus eu
\end{itemize}
\item Aliquam blandit faucibus nisi, sit amet dapibus enim tempus eu
\item Nulla commodo, erat quis gravida posuere, elit lacus lobortis est, quis porttitor odio mauris at libero
\item Nam cursus est eget velit posuere pellentesque
\item Vestibulum faucibus velit a augue condimentum quis convallis nulla gravida
\item Nam cursus est eget velit posuere pellentesque
\end{itemize}
\end{frame}

%------------------------------------------------

\begin{frame}[fragile]
\frametitle{Code Example}

% If using lstlinebgrd package
%		\begin{lstlisting}[language=c,caption=C Code,basicstyle=\ttfamily,linebackgroundcolor={\ifnum\value{lstnumber}=2\color{uni-green}\else\color{white}\fi}]		
	
	{
		
		\begin{lstlisting}[language=c,caption=C Code,basicstyle=\ttfamily]		
int cond_select(bool b, int x, int y){
	if(b){
		return x;
	} 
	else {
		return y;
	}
}
		\end{lstlisting}
		
	}
	
%------------------------------------------------

\end{frame}

\begin{frame}
\frametitle{Blocks of Highlighted Text}
\begin{block}{Block 1}
Lorem ipsum dolor sit amet, consectetur adipiscing elit. Integer lectus nisl, ultricies in feugiat rutrum, porttitor sit amet augue. Aliquam ut tortor mauris. Sed volutpat ante purus, quis accumsan dolor.
\end{block}

\begin{block}{Block 2}
Pellentesque sed tellus purus. Class aptent taciti sociosqu ad litora torquent per conubia nostra, per inceptos himenaeos. Vestibulum quis magna at risus dictum tempor eu vitae velit.
\end{block}

\end{frame}

%------------------------------------------------

\begin{frame}
\frametitle{Multiple Columns}
\begin{columns}[c] % The "c" option specifies centered vertical alignment while the "t" option is used for top vertical alignment

\column{.45\textwidth} % Left column and width
\textbf{Heading}
\begin{enumerate}
\item Statement
\item Explanation
\item Example
\end{enumerate}

\column{.5\textwidth} % Right column and width
\begin{figure}
	\includegraphics[width=1\textwidth]{images/unibuilding.png}
\end{figure}


\end{columns}
\end{frame}

%------------------------------------------------
\section{Second Section}
%------------------------------------------------

\begin{frame}
\frametitle{Table}
\begin{table}
\begin{tabular}{l l l}
\toprule
\textbf{Treatments} & \textbf{Response 1} & \textbf{Response 2}\\
\midrule
Treatment 1 & 0.0003262 & 0.562 \\
Treatment 2 & 0.0015681 & 0.910 \\
Treatment 3 & 0.0009271 & 0.296 \\
\bottomrule
\end{tabular}
\caption{Table caption}
\end{table}
\end{frame}

%------------------------------------------------

\begin{frame}
\frametitle{Theorem}
\begin{theorem}[Mass--energy equivalence]
$E = mc^2$
\end{theorem}
\end{frame}

%------------------------------------------------

\begin{frame}
\frametitle{Figure}
Change the location on this slide to include your own image from the images directory found on the same directory as the template .TeX file. Can also include overlay image in the second frame of this slide.
\begin{figure}
\includegraphics[width=0.8\linewidth]{images/logo.pdf}
\end{figure}

\begin{onlyenv}<2>
	%\begin{tikzpicture}[remember picture,overlay]
	%\node[drop shadow={shadow xshift=.8ex,shadowyshift=-.8ex},fill=white,draw] at (0,0) {\includegraphics[width=12cm]{images/deadstoreex}};
	%%\node at (current page.center) {\includegraphics[width=12cm]{images/deadstoreex}};
	%
	%
	%\end{tikzpicture}
	
	\begin{tikzpicture}[remember picture,overlay]
	\node[drop shadow={shadow xshift=.8ex,shadow yshift=-.8ex},fill=white,draw] at (current page.center) {\includegraphics[width=11cm]{images/logo.pdf}};
	\end{tikzpicture}
	
\end{onlyenv}
\end{frame}




%------------------------------------------------

\begin{frame}[fragile] % Need to use the fragile option when verbatim is used in the slide
\frametitle{Citation}
An example of the \verb|\cite| command to cite within the presentation:\\~

This statement requires citation \cite{p1}.
\end{frame}

%------------------------------------------------

\begin{frame}
\frametitle{References}
\footnotesize{
\begin{thebibliography}{99} % Beamer does not support BibTeX so references must be inserted manually as below
\bibitem[Smith, 2012]{p1} John Smith (2012)
\newblock Title of the publication
\newblock \emph{Journal Name} 12(3), 45 -- 678.
\end{thebibliography}
}
\end{frame}

%------------------------------------------------

\begin{frame}
\Huge{\centerline{The End}}
\end{frame}

%----------------------------------------------------------------------------------------

\end{document} 