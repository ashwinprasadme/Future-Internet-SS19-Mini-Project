%%%%%%%%%%%%%%%%%%%%%%%%%%%%%%%%%%%%%%%%%%%%%%%
% University Paderborn Beamer Presentation 

% Author: Ashwin Prasad Shivarpatna Venkatesh 

%This template is free: you can redistribute it and/or modify
%it under the terms of the GNU General Public License as published by
%the Free Software Foundation, either version 3 of the License, or any later version.
%
%This program is distributed in the hope that it will be useful,
%but WITHOUT ANY WARRANTY; without even the implied warranty of
%MERCHANTABILITY or FITNESS FOR A PARTICULAR PURPOSE.  See the
%GNU General Public License for more details.
%
%You should have received a copy of the GNU General Public License
%along with this program.  If not, see <https://www.gnu.org/licenses/>.

%%%%%%%%%%%%%%%%%%%%%%%%%%%%%%%%%%%%%%%%%%%%%%%

\documentclass{beamer}
% Default page size 12.8cm x 9.6cm

% import packages and user-defined commands
\input{internal/packages}
\input{internal/commands}
\input{internal/upb-unicolors}
\input{internal/upb-theme}
% Title
\title{Complex Service Orchestration with OSM and OpenStack} 

% Sub Title
\subtitle{Mini-Project Intermediate Presentation}

% Your name
\author{Ashwin Prasad Shivarpatna Venkatesh}

% Your institution for the title page
\institute{Future Internet - SS19} 

% Date, can be changed to a custom date
\date{\today} 

% Choose primary UPB color for title, headings etc.. 
% Choose from 
% (uni-cyan, uni-black, uni-blue, uni-orange, uni-purple, uni-green)
\newcommand{\upbcolor}{uni-blue} 




%----------------------------------------------------------------------------------------
%	TITLE PAGE
%----------------------------------------------------------------------------------------


\begin{document}

{
\upbtitlebackground 
\begin{frame}
%\upblogo
\titlepage % Print the title page as the first slide
\end{frame}
}


%----------------------------------------------------------------------------------------
%	PRESENTATION SLIDES
%----------------------------------------------------------------------------------------

\begin{frame}
\frametitle{Goals}
\begin{itemize}
	\item Hands-on experiments with tools\\
	\begin{itemize}
		\item Open Source MANO (OSM) / Sonata
		\item OpenStack
		\item Ubuntu Cloud Image
		\item Cloud-init
		\item ...
	\end{itemize} \pause
	\item Build and deploy a complex service in virtual environment\pause
	\item Get \textbf{Servce Function Chaining:} VNF Forwarding Graph to work (?)	
	\begin{itemize}
		\item Currently unstable in OSM, could lead to a pull request
	\end{itemize}

\end{itemize}
\end{frame}


\begin{frame}
\Huge{\centerline{Scenario}}
\end{frame}


\begin{frame}
\begin{onlyenv}<1>
	
	\begin{figure}
		\centering
		\includegraphics[width=1\linewidth]{images/1}
		\label{fig:1}
	\end{figure}
	
\end{onlyenv}

\begin{onlyenv}<2>
	
	\begin{figure}
		\centering
		\includegraphics[width=1\linewidth]{images/2}
		\label{fig:2}
	\end{figure}
	
\end{onlyenv}

\begin{onlyenv}<3>
	
	\begin{figure}
		\centering
		\includegraphics[width=1\linewidth]{images/3}
		\label{fig:3}
	\end{figure}
	
\end{onlyenv}


\end{frame}


%------------------------------------------------

\begin{frame}
\Huge{\centerline{Teammates?}}
\end{frame}

%----------------------------------------------------------------------------------------

\end{document} 